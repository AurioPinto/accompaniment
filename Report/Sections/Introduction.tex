
\section{The Road to 5G}

\textcolor{black}{In only the past year, preliminary interest and discussions about a possible 5G standard have evolved into a full-fledged conversation that has captured the attention and imagination of researchers and engineers around the world. As the long-term evolution (LTE) system embodying 4G has now been deployed and is reaching maturity, where only incremental improvements and small amounts of new spectrum can be expected, it is natural for researchers to ponder “what’s next?”. However, this is not a mere intellectual exercise. Thanks mainly to the annual visual network index (VNI) reports released by Cisco, we have quantitative evidence that the wireless data explosion is real and will continue. Driven largely by smart phones, tablets, and video streaming, the most recent (Feb. 2014) VNI report and forecast makes plain that an incremental approach will not come close to meeting the demands that networks will face by 2020. In just a decade, the amount of IP data handled by wireless networks will have increased by well over a factor of 100: from under 3 exabytes in 2010 to over 190 exabytes by 2018, on pace to exceed 500 exabytes by 2020. This deluge of data has been driven chiefly by video thus far, but new unforeseen applications can reasonably be expected to materialize by 2020. In addition to the sheer volume of data, the number of devices and the data rates will continue to grow exponentially. The number of devices could reach the tens or even hundreds of billions by the time 5G comes to fruition, due to many new applications beyond personal communications. It is our duty as engineers to meet these intense demands via innovative new technologies that are smart and efficient yet grounded in reality. Academia is engaging in large collaborative projects such as METIS and 5GNOW, while the industry is driving preliminary 5G standardization activities.}

\section{Engineering Requirements for 5G}

In order to more concretely understand the engineering challenges facing 5G, and to plan to meet the requirements, it is necessary to first identify the requirements for a 5G system. The following items are requirements in each key dimension, but it should be stressed that not all of these need to be satisfied simultaneously. Different applications will place different requirements on the performance, and peak requirements that will need to be satisfied in certain configurations are mentioned below. For example, very-high-rate applications such as streaming high-definition video may have relaxed latency and reliability requirements compared to driver less cars or public safety applications, where latency and reliability are paramount but lower data rates can be tolerated.
\begin{flushleft}1) Data Rate: The need to support the mobile data traffic explosion is unquestionably the main driver behind 5G. Data rate can be measured in several different ways, and there will be a 5G goal target for each such metric:\end{flushleft}
\begin{flushleft}a) Aggregate data rate refers to the total amount of data the network can serve, characterized in units of bits/s/area. The general consensus is that this quantity will need to increase by roughly 1000 times from 4G to 5G.\end{flushleft}
\begin{flushleft}b) Edge rate, or 5user can reasonably expect to receive when in range of the network, and so is an important metric and has a concrete engineering meaning. Goals for the 5G edge rate range from 100 Mbps (easily enough to support high-definition streaming) to as much as 1 Gbps. Meeting 100 Mbps for 95be is extraordinarily challenging, even with major technological advances. This requires about 100 times advance since current 4G systems have a typical 5the precise number varies quite widely depending on the load, cell size, and other factors.\end{flushleft}
\begin{flushleft}c) Peak rate is the best-case data rate that a user can hope to achieve under any conceivable network configuration. The peak rate is a marketing number, devoid of much meaning to engineers, but in any case, it will likely be in the range of tens of Gbps. Meeting the requirements in (a) and (b), which are about 1000 times and 100 times the current 4G technology, respectively, are part of the main focus of this research.\end{flushleft}
\begin{flushleft}2) Latency: Current 4G roundtrip latencies are in the order of about 15 ms and are based on the 1 ms subframe time with necessary overheads for resource allocation and access. Although this latency is sufficient for most current services, anticipated 5G applications include two-way gaming, novel cloud-based technologies such as those that may be touchscreen activated (the “tactile Internet”), and virtual and enhanced reality (e.g., Google glass or other wearable computing devices). As a result, 5G will need to be able to support a roundtrip latency of about 1 ms, an order of magnitude faster than 4G. In addition to shrinking down the subframe structure, such severe latency constraints may have important implications on design choices at several layers of the protocol stack and the core network.\end{flushleft}
\begin{flushleft}3) Energy and Cost: As we move to 5G, costs and energy consumption will, ideally, decrease, but at least they should not increase on a per-link basis. Since the per-link data rates being offered will be increasing by about 100 times, this means that the Joules per bit and cost per bit will need to fall by at least 100 times. In this research, we do not address energy and cost in a quantitative fashion, but we are intentionally advocating technological solutions that promise reasonable cost and power scaling. For example, mmWave spectrum should be 10 to 100 times cheaper per Hz than the 3G and 4G spectrum below 3 GHz. Similarly, small cells should be 10 to 100 times cheaper and more power-efficient than macrocells. A major cost consideration for 5G, even more so than in 4G due to the new BS densities and increased bandwidth, is the backhaul from the network edges into the core\end{flushleft}

% \section{Research Problem}
% Lorem ipsum dolor sit amet, consectetur adipiscing elit. Duis ut ipsum nec orci interdum sollicitudin ut eu nunc. Pellentesque ultricies eros in justo sagittis, eget blandit velit aliquet. Aenean ac lectus nibh. Quisque ac est pellentesque, ullamcorper sem sit amet, pharetra quam. Morbi ullamcorper placerat diam, sed tincidunt odio.

% \subsection{Chapter Subheading}
% Lorem ipsum dolor sit amet, consectetur adipiscing elit. Duis ut ipsum nec orci interdum sollicitudin ut eu nunc. Pellentesque ultricies eros in justo sagittis, eget blandit velit aliquet. Aenean ac lectus nibh. Quisque ac est pellentesque, ullamcorper sem sit amet, pharetra quam. Morbi ullamcorper placerat diam, sed tincidunt odio.

% \section{Aim and Scope}
% Lorem ipsum dolor sit amet, consectetur adipiscing elit. Duis ut ipsum nec orci interdum sollicitudin ut eu nunc. Pellentesque ultricies eros in justo sagittis, eget blandit velit aliquet. Aenean ac lectus nibh. Quisque ac est pellentesque, ullamcorper sem sit amet, pharetra quam. Morbi ullamcorper placerat diam, sed tincidunt odio.

% \section{Outline of the Thesis}
% Lorem ipsum dolor sit amet, consectetur adipiscing elit. Duis ut ipsum nec orci interdum sollicitudin ut eu nunc. Pellentesque ultricies eros in justo sagittis, eget blandit velit aliquet. Aenean ac lectus nibh. Quisque ac est pellentesque, ullamcorper sem sit amet, pharetra quam. Morbi ullamcorper placerat diam, sed tincidunt odio.
