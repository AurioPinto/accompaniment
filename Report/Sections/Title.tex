% \documentclass{ctexart}
\begin{titlepage}
\includegraphics{Images/tongji_university.jpg}\\
\vspace{2cm}
Final Report     in Wireless Communication Systems
    \begin{center}
        \vspace*{1cm}
        
        {\large \textbf{Will the World Improve with 5G Technology?}}
        
        \vspace{0.5cm}
       How Beneficial Can It Be?
        
        \vspace{1.5cm}
        by \\
        \vspace{1.5cm}
        Aurio Estefan Augusto Pinto \url{auriopinto2000@tongji.edu.cn}$
                
        \vspace{1.0cm}
    \end{center}

\noindent{
\textcolor{black}{{\bf Abstract} - 5G technology will not be an incremental advance on 4G. The previous four generations of cellular technology have each been a major paradigm shift that has broken backwards compatibility. And indeed, 5G will need to be a paradigm shift that includes very high carrier frequencies with massive bandwidths, extreme base station and device densities as well as unprecedented numbers of antennas. But unlike the previous four generations, it will also be highly integrative: tying any new 5G air interface and spectrum together with LTE and WiFi to provide universal high-rate coverage and a seamless user experience. To support this, the core network will also have to reach unprecedented levels of flexibility and intelligence, spectrum regulation will need to be rethought and improved, and energy and cost efficiency will become even more critical considerations. This research discusses all of these topics, identifying key challenges for future research and preliminary 5G standardization activities}}

\vfill
\textcolor[rgb]{0.5,0.5,0.5}{
    \begin{flushleft}
    { \small
    10052101 \\
    Sitifan - 1656038\\
    Professor: Yin Xuefeng \\
    % Examiner: [Full name] \\
    Word Count: 3000 \\
    }
    \end{flushleft}
}
      
\end{titlepage}
